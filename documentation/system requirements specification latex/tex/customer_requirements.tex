Include a header paragraph specific to your product here. Customer requirements are those required features and functions specified for and by the intended audience for this product. This section establishes, clearly and concisely, the "look and feel" of the product, what each potential end-user should expect the product do and/or not do. Each requirement specified in this section is associated with a specific customer need that will be satisfied. In general Customer Requirements are the directly observable features and functions of the product that will be encountered by its users. Requirements specified in this section are created with, and must not be changed without, specific agreement of the intended customer/user/sponsor.

\subsection{Requirement Name}
\subsubsection{Description}
A detailed description of the feature/function that satisfies the requirement. For example: \textit{The GUI background will be slate blue. This specific color is required in order to ensure that the GUI matches other similar software products offered by the customer. Slate blue is specified as \#007FFF, using six-digit hexadecimal color specification.} It is acceptable and advisable to include drawings/graphics in the description if it aids understanding of the requirement.
\subsubsection{Source}
The source of the requirement (e.g. customer, sponsor, specified team member (by name), federal regulation, local laws, CSE Senior Design project specifications, etc.)
\subsubsection{Constraints}
A detailed description of realistic constraints relevant to this requirement. Economic, environmental, social, political, ethical, health \& safety, manufacturability, and sustainability should be discussed as appropriate.
\subsubsection{Standards}
A detailed description of any specific standards that apply to this requirement (e.g. \textit{NSTM standard xx.xxx.x. color specifications \cite{Rubin2012}}. Standards exist for practically everything (ATC standard fuses, IEEE 802.15.4 embedded wireless, TLS 1.3 encryption, etc.), so be sure that you research and document which ones will be followed in meeting this requirement.
\subsubsection{Priority}
The priority of this requirement relative to other specified requirements. Use the following priorities:
\begin{itemize}
\item Critical (must have or product is a failure)
\item High (very important to customer acceptance, desirability)
\item Moderate (should have for proper product functionality);
\item Low (nice to have, will include if time/resource permits)
\item Future (not feasible in this version of the product, but should be considered for a future release).
\end{itemize}

\subsection{Requirement Name}
\subsubsection{Description}
Detailed requirement description...
\subsubsection{Source}
Source
\subsubsection{Constraints}
Detailed description of applicable constraints...
\subsubsection{Standards}
List of applicable standards
\subsubsection{Priority}
Priority
