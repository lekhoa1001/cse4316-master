This section should contain a list of at least 5 of the most critical risks related to your project. Additionally, the probability of occurrence, size of loss, and risk exposure should be listed. For size of loss, express units as the number of days by which the project schedule would be delayed. For risk exposure, multiply the size of loss by the probability of occurrence to obtain the exposure in days. For example:

The following high-level risk census contains identified project risks with the highest exposure. Mitigation strategies will be discussed in future planning sessions.

\begin{table}[h]
\resizebox{\textwidth}{!}{
\begin{tabular}{|l|l|l|l|}
\hline
 \textbf{Risk description} & \textbf{Probability} & \textbf{Loss (days)} & \textbf{Exposure (days)} \\ \hline
Team communication delays  & 0.20 & 30 & 6  \\ \hline
  App team not learning new software in time  & 0.30 & 15 & 5  \\ \hline
 Changes in user requirements  & 0.60 & 25 & 15  \\ \hline
 EE team not completing shutter functionality  & 0.15 & 10 & 1.5  \\ \hline
Firmware team not fully implementing connectivity  & 0.20 & 5 & 1  \\ \hline
\end{tabular}}
\caption{Overview of highest exposure project risks} 
\end{table}